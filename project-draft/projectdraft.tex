%
% File projectdraft.tex
%
%% Based on the style files for EMNLP 2020, which were
%% Based on the style files for ACL 2020, which were
%% Based on the style files for ACL 2018, NAACL 2018/19, which were
%% Based on the style files for ACL-2015, with some improvements
%%  taken from the NAACL-2016 style
%% Based on the style files for ACL-2014, which were, in turn,
%% based on ACL-2013, ACL-2012, ACL-2011, ACL-2010, ACL-IJCNLP-2009,
%% EACL-2009, IJCNLP-2008...
%% Based on the style files for EACL 2006 by 
%%e.agirre@ehu.es or Sergi.Balari@uab.es
%% and that of ACL 08 by Joakim Nivre and Noah Smith

\documentclass[11pt,a4paper]{article}
\usepackage[hyperref]{projectdraft}
\usepackage{times}
\usepackage{latexsym}
\renewcommand{\UrlFont}{\ttfamily\small}

% This is not strictly necessary, and may be commented out,
% but it will improve the layout of the manuscript,
% and will typically save some space.
\usepackage{microtype}

\aclfinalcopy
%\def\aclpaperid{***} %  Enter the acl Paper ID here

%\setlength\titlebox{5cm}
% You can expand the titlebox if you need extra space
% to show all the authors. Please do not make the titlebox
% smaller than 5cm (the original size); we will check this
% in the camera-ready version and ask you to change it back.

% Content lightly modified from original work by Jesse Dodge and Noah Smith


\newcommand\BibTeX{B\textsc{ib}\TeX}

\title{Project Draft for Semi-Supervised Classification with Graph Convolutional Networks}

\author{Zhisheng Xu and Goutham Pakalapati\\
  \texttt{\{zx41, gpakal2\}@illinois.edu}
  \\[2em]
  Group ID: 28\\
  Paper ID: 11\\
  Presentation link: \url{N/A} \\
  Code link: \url{N/A}
}

\begin{document}
\maketitle

% All sections are mandatory.
% Keep in mind that your page limit is 8, excluding references.
% For specific grading rubrics, please see the project instruction.

\section{Introduction}
% A  few  sentences  placing  the  work  in  context. Limit it to a few paragraphs at most; your report is on reproducing a piece of work, you don’t have to motivate that work. However, it should be clear enough what the original paper is about and what the contribution of it is.

The Semi-Supervised Classification with Graph Convolutional Networks \cite{kipf2017semi} aims to solve the problem of semi-supervised classification on graph-structured data. GCNs are engineered to take advantage of the data's local structure within graphs by utilizing convolutional layers on the adjacency matrix of the graph. It develops a scalable and efficient method that can accurately classify nodes or predict labels for unseen data points in the graph which has limited amount of labeled data.

\section{Scope of reproducibility}

% Explain the claims from the paper you picked for the reproduction study and briefly motivate your choice. We recommend picking the claim that is the central contribution of the paper. To find what this contribution is, try to summarize the most important result of the paper in 1-2 sentences, e.g. ``This paper introduces a new activation function X that outperforms a similar activation function Y on tasks Z,V,W.'' 

% Make the scope as specific as possible. It should be something that can be supported or rejected by your data. For example, consider this scope: 

% \textit{``Contextual embedding models have shown strong performance on a number of tasks across NLP. We will run experiments evaluating two types of contextual embedding models on datasets X, Y, and Z.''}


% That scope is too broad and lacks precise outcome (what is ``strong performance?''). Now consider this scope:

% \textit{``Finetuning pretrained BERT on SST-2 will have higher accuracy than an LSTM trained with GloVe embeddings.''}

% This scope is better because it's more specific and has an outcome that can be either supported or rejected based on your work: 

The proposed GCN model exhibits robust classification performance in terms of both accuracy and speed on benchmark datasets, notably the Cora, Citeseer, and Pubmed citation datasets.

\subsection{Addressed claims from the original paper}

Clearly itemize the claims you are testing:
\begin{itemize}
    \item Accuracy (TODO)
    \item Performance (TODO)
\end{itemize}


\section{Methodology}

% In this section you explain your approach -- did you use the author's code, did you aim to re-implement the approach from the paper description? Summarize the resources (code, documentation, GPUs) that you used. 

To reproduce the results, we will utilize both the datasets and code from the original author. The code has undergone minor modifications to make it compatible with the latest TensorFlow and Scipy packages. Additionally, we plan to further tweak the code to test different activation functions later in the project. The code was obtained from the author's GitHub repository, located at https://github.com/tkipf/gcn. By following the instructions provided in the README file, we were able to successfully execute the models using the datasets provided. The computations were performed on an RTX 3080 Ti GPU.

\subsection{Model descriptions}
% Describe the models used in the original paper, including the architecture, learning objective and the number of parameters.

The GCN model is a two-layer neural network of graph convolutional layers. The first layer is a graph convolutional layer with 16 hidden units followed by a ReLU (Rectified Linear Unit) activation function. The second layer is another graph convolutional layer, which serves as the output layer and has as many output units as the number of classes in the classification problem.

\begin{equation}
  \mathcal{L} = -\sum_{l\in \mathcal{Y}_L}\sum_{f=1}^{F} Y_{lf}\ln Z_{lf}
\end{equation}

\subsection{Data descriptions}
% Describe the datasets you used (with some statistics) and how you obtained them. 

We will be using 3 citation network datasets -- Citeseer, Cora and Pubmed -- nodes are documents and edges are citation links. These datasets are obtained from the author's GitHub repository. Following table contains statistics of the datasets.

\begin{table}[h]
\centering
\begin{tabular}{|c|c|c|c|c|}
\hline
Dataset & Nodes & Edges & Features & Classes \\ \hline
Citeseer & 3,327 & 4,732 & 3,703 & 6 \\ \hline
Cora & 2,708 & 5,429 & 1,433 & 7 \\ \hline
Pubmed & 19,717 & 44,338 & 500 & 3 \\ \hline
\end{tabular}
\caption{Dataset statistics}
\end{table}

\subsection{Hyperparameters}
% Describe how you set the hyperparameters and what the source was for their value (e.g. paper, code or your guess). 

The model is trained using the Adam optimizer with a learning rate of 0.01, a weight decay of 5e-4, and a dropout rate of 0.5. The model is trained for 200 epochs with early stopping after 10 epochs with no improvement in validation loss.

Learning rate: 0.01
Weight decay (L2 regularization): 5e-4
Dropout rate: 0.5
Number of epochs: 200
Early stopping: After 10 epochs with no improvement in validation loss
Optimization algorithm: Adam

\subsection{Implementation}
% Describe whether you use the existing code or write your own code, with the link to the code. Note that the github repo you link to should be public and have a clear documentation.



\subsection{Computational requirements}
% Provide information on computational requirements for each of your experiments. For example, the number of CPU/GPU hours and memory requirements.
% Mention both your estimation made before running the experiments (i.e. in the proposal) and the actual resources you used to reproducing the experiments. 
% \textbf{\textit{You'll need to think about this ahead of time, and write your code in a way that captures this information so you can later add it to this section.} }

The author used a 16-core Intel Xeon CPU E5-2640 v3 @ 2.60GHz with a NVIDIA GeForce GTX TITAN X. We will be running the model with AMD Ryzen 7 3700X 8-Core Processor and NVIDIA GeForce RTX 3080 Ti. In the Results section of the original paper, it states the running time on each of the datasets:

\begin{itemize}
  \item Citeseer: 7 seconds
  \item Cora: 4 seconds
  \item Pubmed: 38 seconds
\end{itemize}

Since the power of our GPU is greater than theirs, we expect the running time to be faster than the original paper.

\section{Results}
% Start with a high-level overview of your results. Does your work support the claims you listed in section 2.1? Keep this section as factual and precise as possible, reserve your judgement and discussion points for the next ``Discussion'' section. 

% Go into each individual result you have, say how it relates to one of the claims, and explain what your result is. Logically group related results into sections. Clearly state if you have gone beyond the original paper to run additional experiments and how they relate to the original claims. 

% Tips 1: Be specific and use precise language, e.g. ``we reproduced the accuracy to within 1\% of reported value, that upholds the paper's conclusion that it performs much better than baselines.'' Getting exactly the same number is in most cases infeasible, so you'll need to use your judgement call to decide if your results support the original claim of the paper. 

% Tips 2: You may want to use tables and figures to demonstrate your results.

% The number of subsections for results should be the same as the number of hypotheses you are trying to verify.

\begin{tabular}{|c|c|c|}
  \hline
  \textbf{Dataset} & \textbf{Accuracy} & \textbf{Runtime (s)} \\
  \hline
  Citeseer & 70.6\% & 8.2 \\
  Cora & 81.7\% & 3.4 \\
  Pubmed & 79.4\% & 12.5 \\
  \hline
\end{tabular}

\subsection{Result 1}

\subsection{Result 2}

\subsection{Additional results not present in the original paper}

% Describe any additional experiments beyond the original paper. This could include experimenting with additional datasets, exploring different methods, running more ablations, or tuning the hyperparameters. For each additional experiment, clearly describe which experiment you conducted, its result, and discussions (e.g. what is the indication of the result).

\section{Plans}

-We could use something like grid search on other datasets as well to further optimize some of the hyperparameters used in the model.
-Processing the graph before feeding into a GCN (e.g. adding new edges based on similarity measure)

\bibliographystyle{draft_natbib}
\bibliography{projectdraft}

%\appendix

\end{document}
